\chapter{Arhitektura i dizajn sustava}
		
		\textbf{\textit{dio 1. revizije}}\\

		\textit{ Potrebno je opisati stil arhitekture te identificirati: podsustave, preslikavanje na radnu platformu, spremišta podataka, mrežne protokole, globalni upravljački tok i sklopovsko-programske zahtjeve. Po točkama razraditi i popratiti odgovarajućim skicama:}
	\begin{itemize}
		\item 	\textit{izbor arhitekture temeljem principa oblikovanja pokazanih na predavanjima (objasniti zašto ste baš odabrali takvu arhitekturu)}
		\item 	\textit{organizaciju sustava s najviše razine apstrakcije (npr. klijent-poslužitelj, baza podataka, datotečni sustav, grafičko sučelje)}
		\item 	\textit{organizaciju aplikacije (npr. slojevi frontend i backend, MVC arhitektura) }		
	\end{itemize}

	
		

		

				
		\section{Baza podataka}
			
			\textbf{\textit{dio 1. revizije}}\\
			
		{Koristimo relacijsku bazu podataka s implementacijom pomoću DBML-a. DBML (Database Markup Language) je jezik označavanja koji omogućuje deklarativno opisivanje relacijskih baza podataka. Time možemo jednostavno definirati strukture baze podataka pomoću tekstualnog zapisa. Glavne komponente naše baze podataka su:}
				\begin{itemize}
					\item 	\textbf{documents}
					\item 	\textbf{users}
					\item 	\textbf{signatures}
					\item 	\textbf{audits}
					\item 	\textbf{archive}
					\item 	\textbf{image}
					\item 	\textbf{user\textunderscore role}
					\item	\textbf{roles}
				\end{itemize}
		
			\subsection{Opis tablica}
			
				
				\textbf{Documents} 
				{  Sadrži informacije o dokumentima, uključujući vrstu dokumenta, sliku dokumenta, sažetak, status, vlasnika, oznaku o ispravnom skeniranju i vremensku oznaku skeniranja. Ovaj entitet u vezi je \textit{One-to-Many} s entitetom \textbf{users} preko atributa \textit{ownerID}, u vezi je \textit{One-to-Many} s entitetom \textbf{signatures} preko atributa \textit{documentID}, u vezi je \textit{One-to-Many} s entitetom \textbf{audits} preko atributa \textit{documentID}, u vezi je \textit{One-to-Many} s entitetom \textbf{archive} preko atributa \textit{documentID}, u vezi je \textit{One-to-One} s entitetom \textbf{image} preko atributa \textit{imageID}, u vezi je \textit{Many-to-One} s entitetom \textbf{users} preko atributa \textit{ownerID}.}
				
				
				\begin{longtblr}[
					label=none,
					entry=none
					]{
						width = \textwidth,
						colspec={|X[8,l]|X[5, l]|X[20, l]|}, 
						rowhead = 1,
					} %definicija širine tablice, širine stupaca, poravnanje i broja redaka naslova tablice
					\hline \SetCell[c=3]{c}{\textbf{documents}}	 \\ \hline[3pt]
					\SetCell{LightGreen}documentID & INT & identifikacijski broj dokumenta  	\\ \hline
					documentType	& ENUM & tip skeniranog dokumenta: račun, ponuda ili interni	\\ \hline 
					\SetCell{LightBlue}imageID & INT & identifikacijski broj slike dokumenta  \\ \hline 
					summary & TEXT	& tekst koji sadrži skenirani dokument 		\\ \hline 
					status	& ENUM & trenutni status dokumenta: odobren, odbijen, potpisan, reviziran, potpisan i arhiviran ili arhiviran 	\\ \hline 
					correctlyScanned & BOOL & je li dokument označen kao ispravno skeniran ili nije \\ \hline
					\SetCell{LightBlue}ownerID & INT & identifikacijski broj korisnika koji je skenirao dokument \\ \hline
					scanTimestamp & DATETIME & oznaka datuma i vremena skeniranja dokumenta \\ \hline
				\end{longtblr}
				
				\textbf{Users} 
				{  Sprema informacije o korisnicima, uključujući korisničko ime i lozinku. Ovaj entitet u vezi je \textit{One-to-Many} s entitetom \textbf{documents} preko atributa \textit{userID}, u vezi je \textit{One-to-Many} s entitetom \textbf{signatures} preko atributa \textit{userID}, u vezi je \textit{One-to-Many} s entitetom \textbf{audits} preko atributa \textit{userID}, u vezi je \textit{One-to-Many} s entitetom \textbf{archive} preko atributa \textit{userID}, u vezi je \textit{One-to-One} s entitetom \textbf{user\textunderscore role} preko atributa \textit{userID}.}
				
				\begin{longtblr}[
					label=none,
					entry=none
					]{
						width = \textwidth,
						colspec={|X[8,l]|X[5, l]|X[20, l]|}, 
						rowhead = 1,
					} %definicija širine tablice, širine stupaca, poravnanje i broja redaka naslova tablice
					\hline \SetCell[c=3]{c}{\textbf{users}}	 \\ \hline[3pt]
					\SetCell{LightGreen}userID & INT & identifikacijski broj korisnika  	\\ \hline
					username	& VARCHAR & korisničko ime za prijavu korisnika	\\ \hline 
					password & VARCHAR & šifa za prijavu korisnika  \\ \hline 
				\end{longtblr}
				
				\textbf{Signatures}
				{  Služi za praćenje potpisa na dokumentima, s informacijama o korisniku koji je potpisao, statusu potpisa i vremenskoj oznaci potpisa. Ovaj entitet u vezi je \textit{Many-to-One} s entitetom \textbf{users} preko atributa \textit{signedBy}, u vezi je \textit{One-to-One} s entitetom \textbf{documents} preko atributa \textit{documentID}.}
				
				\begin{longtblr}[
					label=none,
					entry=none
					]{
						width = \textwidth,
						colspec={|X[8,l]|X[5, l]|X[20, l]|}, 
						rowhead = 1,
					} %definicija širine tablice, širine stupaca, poravnanje i broja redaka naslova tablice
					\hline \SetCell[c=3]{c}{\textbf{signatures}}	 \\ \hline[3pt]
					\SetCell{LightGreen}signatureID & INT & identifikacijski broj potpisanog dokumenta  	\\ \hline
					\SetCell{LightBlue}documentID	& INT & identifikacijski broj dokumenta	\\ \hline 
					\SetCell{LightBlue}signedBy & INT & identifikacijski broj korisnika koji je potpisao ili treba potpisati dokument  \\ \hline 
					status & ENUM & status dokumenta za potpisivanje: potpisan ili čeka na potpisivanje \\ \hline
					signedAt & DATETIME & oznaka datuma i vremena potpisivanja dokumenta \\ \hline
				\end{longtblr}
				
				\textbf{Audits}
				{  Sadrži podatke o revizijama dokumenata, uključujući korisnika koji je obavio reviziju, status revizije i vremensku oznaku. Ovaj entitet u vezi je \textit{Many-to-One} s entitetom \textbf{users} preko atributa \textit{auditedBy}, u vezi je \textit{One-to-One} s entitetom \textbf{documents} preko atributa \textit{documentID}.}
				
				\begin{longtblr}[
					label=none,
					entry=none
					]{
						width = \textwidth,
						colspec={|X[8,l]|X[5, l]|X[20, l]|}, 
						rowhead = 1,
					} %definicija širine tablice, širine stupaca, poravnanje i broja redaka naslova tablice
					\hline \SetCell[c=3]{c}{\textbf{audits}}	 \\ \hline[3pt]
					\SetCell{LightGreen}auditID & INT & identifikacijski broj reviziranog dokumenta  	\\ \hline
					\SetCell{LightBlue}documentID	& INT & identifikacijski broj dokumenta	\\ \hline 
					\SetCell{LightBlue}auditedBy & INT & identifikacijski broj korisnika koji je revizirao ili treba revizirati dokument  \\ \hline 
					status & ENUM & status dokumenta za reviziju: reviziran ili čeka na reviziranje \\ \hline
					auditedAt & DATETIME & oznaka datuma i vremena reviziranja dokumenta \\ \hline
				\end{longtblr}
				
				\textbf{Archive}
				{  Koristi se za praćenje arhiviranja dokumenata, s informacijama o korisniku koji je arhivirao dokument, statusu arhiviranja, vremenskoj oznaci i broju arhive. Ovaj entitet u vezi je \textit{One-to-One} s entitetom \textbf{documents} preko atributa \textit{documentID}, u vezi je \textit{Many-to-One} s entitetom \textbf{users} preko atributa \textit{archiveBy}.}
				
				\begin{longtblr}[
					label=none,
					entry=none
					]{
						width = \textwidth,
						colspec={|X[8,l]|X[5, l]|X[20, l]|}, 
						rowhead = 1,
					} %definicija širine tablice, širine stupaca, poravnanje i broja redaka naslova tablice
					\hline \SetCell[c=3]{c}{\textbf{archive}}	 \\ \hline[3pt]
					\SetCell{LightGreen}archiveNumber & INT & identifikacijski broj arhiviranog dokumenta  	\\ \hline
					\SetCell{LightBlue}documentID	& INT & identifikacijski broj dokumenta	\\ \hline 
					\SetCell{LightBlue}archiveBy & INT & identifikacijski broj korisnika koji je arhivirao ili treba arhivirati dokument  \\ \hline 
					status & ENUM & status dokumenta za arhiviranje: arhiviran ili čeka na arhiviranje \\ \hline
					archivedAt & DATETIME & oznaka datuma i vremena arhiviranja dokumenta \\ \hline
				\end{longtblr}
				
				\textbf{Image}
				{  Sprema slike dokumenata. Ovaj entitet u vezi je \textit{One-to-One} s entitetom \textbf{documents} preko atributa \textit{imageID}.}
				
				\begin{longtblr}[
					label=none,
					entry=none
					]{
						width = \textwidth,
						colspec={|X[8,l]|X[5, l]|X[20, l]|}, 
						rowhead = 1,
					} %definicija širine tablice, širine stupaca, poravnanje i broja redaka naslova tablice
					\hline \SetCell[c=3]{c}{\textbf{image}}	 \\ \hline[3pt]
					\SetCell{LightGreen}imageID & INT & identifikacijski broj slike dokumenta  	\\ \hline
					image & BLOB & spremljena slika dokumenta \\ \hline
				\end{longtblr}
				
				\textbf{Roles}
				{TODO}
				
				\begin{longtblr}[
					label=none,
					entry=none
					]{
						width = \textwidth,
						colspec={|X[8,l]|X[5, l]|X[20, l]|}, 
						rowhead = 1,
					} %definicija širine tablice, širine stupaca, poravnanje i broja redaka naslova tablice
					\hline \SetCell[c=3]{c}{\textbf{roles}}	 \\ \hline[3pt]
					\SetCell{LightGreen}roleID & INT & identifikacijski broj kategorije korisnika  	\\ \hline
					roleName & VARCHAR & ime kategorije korisnika \\ \hline
				\end{longtblr}
				
				\textbf{User\textunderscore role}
				{TODO}
				
				\begin{longtblr}[
					label=none,
					entry=none
					]{
						width = \textwidth,
						colspec={|X[8,l]|X[5, l]|X[20, l]|}, 
						rowhead = 1,
					} %definicija širine tablice, širine stupaca, poravnanje i broja redaka naslova tablice
					\hline \SetCell[c=3]{c}{\textbf{user\textunderscore role}}	 \\ \hline[3pt]
					\SetCell{LightGreen}userID & INT & identifikacijski broj korisnika  	\\ \hline
					\SetCell{LightGreen}roleID & INT & identifikacijski broj kategorije korisnika \\ \hline
				\end{longtblr}
				
				
			
			\subsection{Dijagram baze podataka}
				\textit{ U ovom potpoglavlju potrebno je umetnuti dijagram baze podataka. Primarni i strani ključevi moraju biti označeni, a tablice povezane. Bazu podataka je potrebno normalizirati. Podsjetite se kolegija "Baze podataka".}
			
			\eject
			
			
		\section{Dijagram razreda}
		
			\textit{Potrebno je priložiti dijagram razreda s pripadajućim opisom. Zbog preglednosti je moguće dijagram razlomiti na više njih, ali moraju biti grupirani prema sličnim razinama apstrakcije i srodnim funkcionalnostima.}\\
			
			\textbf{\textit{dio 1. revizije}}\\
			
			\textit{Prilikom prve predaje projekta, potrebno je priložiti potpuno razrađen dijagram razreda vezan uz \textbf{generičku funkcionalnost} sustava. Ostale funkcionalnosti trebaju biti idejno razrađene u dijagramu sa sljedećim komponentama: nazivi razreda, nazivi metoda i vrste pristupa metodama (npr. javni, zaštićeni), nazivi atributa razreda, veze i odnosi između razreda.}\\
			
			\textbf{\textit{dio 2. revizije}}\\			
			
			\textit{Prilikom druge predaje projekta dijagram razreda i opisi moraju odgovarati stvarnom stanju implementacije}
			
			
			
			\eject
		
		\section{Dijagram stanja}
			
			
			\textbf{\textit{dio 2. revizije}}\\
			
			\textit{Potrebno je priložiti dijagram stanja i opisati ga. Dovoljan je jedan dijagram stanja koji prikazuje \textbf{značajan dio funkcionalnosti} sustava. Na primjer, stanja korisničkog sučelja i tijek korištenja neke ključne funkcionalnosti jesu značajan dio sustava, a registracija i prijava nisu. }
			
			
			\eject 
		
		\section{Dijagram aktivnosti}
			
			\textbf{\textit{dio 2. revizije}}\\
			
			 \textit{Potrebno je priložiti dijagram aktivnosti s pripadajućim opisom. Dijagram aktivnosti treba prikazivati značajan dio sustava.}
			
			\eject
		\section{Dijagram komponenti}
		
			\textbf{\textit{dio 2. revizije}}\\
		
			 \textit{Potrebno je priložiti dijagram komponenti s pripadajućim opisom. Dijagram komponenti treba prikazivati strukturu cijele aplikacije.}