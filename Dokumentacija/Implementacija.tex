\chapter{Implementacija i korisničko sučelje}
		
		
		\section{Korištene tehnologije i alati}
		
			 {Za komunikaciju unutar tima koristili smo \textbf{Whatsapp}\footnote{\url{https://www.whatsapp.com/}} i \textbf{Discord}\footnote{\url{https://discord.com/}}. Za izradu UML dijagrama koristili smo: \textbf{dbdiagram}\footnote{\url{https://dbdiagram.io/}} za dijagram baze podataka, \textbf{Visual Paradigm Online}\footnote{\url{https://online.visual-paradigm.com/}} za dijagram razreda te \textbf{Astah}\footnote{\url{https://astah.net/}} za ostale UML dijagrame. Kao sustav upravljanja verzijama je korišten \textbf{Git}\footnote{\url{https://git-scm.com/}} te su naši repozitoriji bili dostupni na \textbf{Githubu}\footnote{\url{https://github.com/}} udaljenom repozitoriju.
			 	
		 	Kao razvojno okruženje je korišten \textbf{PyCharm}\footnote{\url{https://www.jetbrains.com/pycharm/}} za razvoj API-ja te \textbf{Android Studio}\footnote{\url{https://developer.android.com/studio}} za razvoj mobilne aplikacije. Pycharm je razvojno okruženje za python, napravljeno od JetBrains-a te Android Studio je modificirana inačica drugog JetBrains razvojnog okruženja (IntelliJ, razvojno okruženje za Javu), prilagođena razvoju android mobilnih aplikacija.
		 	
		 	Mobilna aplikacija je napravljena s pomoću \textbf{Flutter}\footnote{\url{https://flutter.dev/}} razvojnog okvira koji koristi \textbf{Dart}\footnote{\url{https://dart.dev/}} programski jezik. Flutter i Dart su projekti razvijeni od strane Googlea. Flutter je razvojni okvir za izradu aplikacija na različitim platformama, vrlo popularan zbog mogućnosti kreiranja aplikacije za više platforma paralelno. Dart je moderni programski jezik koji se primarno koristi za razvoj Flutter aplikacija.
		 	API je kreiran s pomoću \textbf{Pythona}\footnote{\url{https://www.python.org/}} te \textbf{FastAPI}\footnote{\url{https://fastapi.tiangolo.com/}} razvojnog okvira. FastAPI je novije razvojno okruženje bazirano na Flasku, vrlo popularnom python razvojnom okruženju, s mogućnosti paralelnog izvođenja zahtjeva te automatskom dokumentacijom po OpenAPI standardu.
		 	
		 	Aplikacija je puštena u pogon na \textbf{Renderu}\footnote{\url{https://render.com/}}, te je korišten \textbf{Microsoft Azure}\footnote{\url{https://azure.microsoft.com/}}.
	 	 	za spremanje slika, za prepoznavanje teksta na slikama te se tamo nalazi naša baza podataka.
			 }
			 		
			\eject 
		
	
		\section{Ispitivanje programskog rješenja}
			
			\textbf{\textit{dio 2. revizije}}\\
			
			 \textit{U ovom poglavlju je potrebno opisati provedbu ispitivanja implementiranih funkcionalnosti na razini komponenti i na razini cijelog sustava s prikazom odabranih ispitnih slučajeva. Studenti trebaju ispitati temeljnu funkcionalnost i rubne uvjete.}
	
			
			\subsection{Ispitivanje komponenti}
			\textit{Potrebno je provesti ispitivanje jedinica (engl. unit testing) nad razredima koji implementiraju temeljne funkcionalnosti. Razraditi \textbf{minimalno 6 ispitnih slučajeva} u kojima će se ispitati redovni slučajevi, rubni uvjeti te izazivanje pogreške (engl. exception throwing). Poželjno je stvoriti i ispitni slučaj koji koristi funkcionalnosti koje nisu implementirane. Potrebno je priložiti izvorni kôd svih ispitnih slučajeva te prikaz rezultata izvođenja ispita u razvojnom okruženju (prolaz/pad ispita). }
			
			
			
			\subsection{Ispitivanje sustava}
			
			 \textit{Potrebno je provesti i opisati ispitivanje sustava koristeći radni okvir Selenium\footnote{\url{https://www.seleniumhq.org/}}. Razraditi \textbf{minimalno 4 ispitna slučaja} u kojima će se ispitati redovni slučajevi, rubni uvjeti te poziv funkcionalnosti koja nije implementirana/izaziva pogrešku kako bi se vidjelo na koji način sustav reagira kada nešto nije u potpunosti ostvareno. Ispitni slučaj se treba sastojati od ulaza (npr. korisničko ime i lozinka), očekivanog izlaza ili rezultata, koraka ispitivanja i dobivenog izlaza ili rezultata.\\ }
			 
			 \textit{Izradu ispitnih slučajeva pomoću radnog okvira Selenium moguće je provesti pomoću jednog od sljedeća dva alata:}
			 \begin{itemize}
			 	\item \textit{dodatak za preglednik \textbf{Selenium IDE} - snimanje korisnikovih akcija radi automatskog ponavljanja ispita	}
			 	\item \textit{\textbf{Selenium WebDriver} - podrška za pisanje ispita u jezicima Java, C\#, PHP koristeći posebno programsko sučelje.}
			 \end{itemize}
		 	\textit{Detalji o korištenju alata Selenium bit će prikazani na posebnom predavanju tijekom semestra.}
			
			\eject 
		
		
		\section{Dijagram razmještaja}
			
			\textbf{\textit{dio 2. revizije}}
			
			 \textit{Potrebno je umetnuti \textbf{specifikacijski} dijagram razmještaja i opisati ga. Moguće je umjesto specifikacijskog dijagrama razmještaja umetnuti dijagram razmještaja instanci, pod uvjetom da taj dijagram bolje opisuje neki važniji dio sustava.}
			
			\eject 
		
		\section{Upute za puštanje u pogon}
		
			\textbf{\textit{dio 2. revizije}}\\
		
			 \textit{U ovom poglavlju potrebno je dati upute za puštanje u pogon (engl. deployment) ostvarene aplikacije. Na primjer, za web aplikacije, opisati postupak kojim se od izvornog kôda dolazi do potpuno postavljene baze podataka i poslužitelja koji odgovara na upite korisnika. Za mobilnu aplikaciju, postupak kojim se aplikacija izgradi, te postavi na neku od trgovina. Za stolnu (engl. desktop) aplikaciju, postupak kojim se aplikacija instalira na računalo. Ukoliko mobilne i stolne aplikacije komuniciraju s poslužiteljem i/ili bazom podataka, opisati i postupak njihovog postavljanja. Pri izradi uputa preporučuje se \textbf{naglasiti korake instalacije uporabom natuknica} te koristiti što je više moguće \textbf{slike ekrana} (engl. screenshots) kako bi upute bile jasne i jednostavne za slijediti.}
			
			
			 \textit{Dovršenu aplikaciju potrebno je pokrenuti na javno dostupnom poslužitelju. Studentima se preporuča korištenje neke od sljedećih besplatnih usluga: \href{https://aws.amazon.com/}{Amazon AWS}, \href{https://azure.microsoft.com/en-us/}{Microsoft Azure} ili \href{https://www.heroku.com/}{Heroku}. Mobilne aplikacije trebaju biti objavljene na F-Droid, Google Play ili Amazon App trgovini.}
			
			
			\eject 