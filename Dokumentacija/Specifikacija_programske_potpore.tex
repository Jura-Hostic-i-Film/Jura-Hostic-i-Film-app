\chapter{Specifikacija programske potpore}
		
	\section{Funkcionalni zahtjevi}
			
			\noindent \textbf{Dionici:}
			
			\begin{packed_enum}
				
				\item Vlasnik (naručitelj)
				\item Članovi organizacije
				
				\begin{packed_enum}
					\item Zaposlenik
					\item Revizor
					\item Računovođa
					\item Direktor
				\end{packed_enum}
				
				\item Administrator
				\item Razvojni tim
				
			\end{packed_enum}
			
			\noindent \textbf{Aktori i njihovi funkcionalni zahtjevi:}
			
			
			\begin{packed_enum}
				
				\item  \underbar{Neregistrirani/neprijavljeni korisnik (inicijator) može:}
				\begin{packed_enum}
					
					\item se prijaviti u sustav pomoću kredencijala (e-mail adresa i lozinka) koje je dobio od superriornije osobe 
					
				\end{packed_enum}
			
				\item  \underbar{Zaposlenik (inicijator) može:}
				\begin{packed_enum}
					
					\item prijaviti se u sustav
					\item pregledati svoje osobne podatke (ime, prezime, nadređeni revizor)
					\item skenirati dokumente koje se spremaju u njegovu internu bazu
					\item potvrditi dokumente koje je prethodno skenirao
					\item poslati revizirane dokumente od njegove strane na dodatan pregled
					\item vidjeti svoju internu bazu prethodno skeniranih dokumenata
					
				\end{packed_enum}
				
				\item  \underbar{Revizor (inicijator) može:}
				\begin{packed_enum}
					
					\item prijaviti se u sustav
					\item pregledati svoje osobne podatke (ime, prezime, popis podređenih zaposlenika)
					\item dobiti obavijesti o skeniranim dokumentima od strane podređenog zaposlenika
					\item skenirati dokumente koje se spremaju u njegovu internu bazu
					\item revizirati primljene dokumente
					\item proslijediti određenom računovođi dokument skeniran i potvrđen od strane podređenog zaposlenika
					
				\end{packed_enum}
				
				\item  \underbar{Računovođa (inicijator) može:}
				\begin{packed_enum}
					
					\item prijaviti se u sustav
					\item pregledati svoje osobne podatke (ime, prezime, vrstu dokumenta za koju je zaslužan)
					\item pregledati dokumente spremne za arhiviranje
					\item arhivirati dokumente
					\item pregledati povijest dokumenata koje je arhivirao u svojoj internoj bazi
					\item prosljediti dokumente direktoru na potpis prije dokumentiranja 
					
				\end{packed_enum}
				
				\item  \underbar{Direktor (inicijator) može:}
				\begin{packed_enum}
					
					\item prijaviti se u sustav
					\item pregledati svoje osobne podatke (ime, prezime, ime tvrtke za koju je zaslužan)
					\item primati obavijesti o zatraženim potpisima o računovođi
					\item pregledati popis svih korisnika aplikacije
					\item pregledati statistike zaposlenika (broj skeniranih dokumenata,
					prosjek skeniranih dokumenata svih zaposlenika te ostale statistike)
					\item promjeniti uloge i ostale podatke o svim ostalim korisnicima unutar tvrtke 
					\item brisati račune otpuštenih zaposlenika tvrtke te ih također stvarati 
					\item pregledati sve dokumente i njihovu povijest(tko ih je skenirao, 
					revizirao i koji računovođa je bio zaslužan za njihovo arhiviranje)
					\item objaviti određeni dokument na društvenim mrežama
					
				\end{packed_enum}
				
				\item  \underbar{Administrator (inicijator) može:}
				\begin{packed_enum}
					
					\item prijaviti se u sustav
					\item dodavati različite tvrtke i njihove direktore
					\item pregledati osobne podatke svih korisnika aplikacije
					\item upozoravati direktora na kršenje uvijete poslovanja aplikacije
					\item brisati korisnika koji krše uvijete poslovanja aplikacije
					
				\end{packed_enum}
				
				\item  \underbar{Baza podataka (sudionik) može:}
				\begin{packed_enum}
					
					\item pohraniti podatke o svim korisnicima aplikacije
					\item pohraniti sve podatke o tvrtkama te arhivitra sve dokumente određene tvrtke
					
				\end{packed_enum}
				
				
			\end{packed_enum}
			
			\eject 
			
			
				
			\subsection{Obrasci uporabe}
							
				\subsubsection{Opis obrazaca uporabe}

					\noindent \underbar{\textbf{UC1 -Registracija u sustav}}
					\begin{packed_item}
	
						\item \textbf{Glavni sudionik: }Direktor, Administrator
						\item  \textbf{Cilj:} Stvoriti korisnički račun za pristup sustavu
						\item  \textbf{Sudionici:} Baza podataka
						\item  \textbf{Preduvjet:} -
						\item  \textbf{Opis osnovnog tijeka:}
						
						\item[] \begin{packed_enum}
	
							\item Aktor odabire opciju za registraciju
							\item Aktor unosi osobne podatke za novog korisnika
							\item Šalje se mail novododanom korisniku s njegovim podacima za prijavu
						\end{packed_enum}
						
						\item  \textbf{Opis mogućih odstupanja:}
						
						\item[] \begin{packed_item}
	
							\item[2.a] Korisnik je već registriran u sustav
							\item[] \begin{packed_enum}
								
								\item Sustav daje audio-vizualnu obavjest aktoru da je korisnik već registriran
								
							\end{packed_enum}
							
							\item[2.b] Neispravna e-mail adresa
							\item[] \begin{packed_enum}
								
								\item Sustav daje auditornu i vizualnu obavjest aktoru da je e-mail adresa neispravna
								\item Aktor upisuje ispravnu e-mail adresu
							\end{packed_enum}
							
						\end{packed_item}
					\end{packed_item}
					
					\noindent \underbar{\textbf{UC2 -Prijava u sustav}}
					\begin{packed_item}
						
						\item \textbf{Glavni sudionik: }članovi organizacije, Administrator
						\item  \textbf{Cilj:} Prijava za pristup funkcionalnost aplikacije
						\item  \textbf{Sudionici:} Baza podataka
						\item  \textbf{Preduvjet:} Aktor je registriran
						\item  \textbf{Opis osnovnog tijeka:}
						
						\item[] \begin{packed_enum}
							
							\item Aktor unosi svoje osobne podatke
							\item Sustav provjerava jesu li podnešeni podaci ispravni
							\item Aktor biva prijavljen u aplikaciju
						\end{packed_enum}
						
						\item  \textbf{Opis mogućih odstupanja:}
						
						\item[] \begin{packed_item}
							
							\item[2.a] Unešeni osobni podaci ne odgovaraju niti jednom registriranom korisniku
							\item[] \begin{packed_enum}
								
								\item Sustav daje audio-vizualnu obavjest aktoru da su osobni podaci neispravni
								\item Sustav daje priliku aktoru za ponovnu prijavu
								
							\end{packed_enum}
							
						\end{packed_item}
					\end{packed_item}
				
					\noindent \underbar{\textbf{UC3 -Brisanje korisnika iz sustava}}
					\begin{packed_item}
						
						\item \textbf{Glavni sudionik: }Direktor, Administrator
						\item  \textbf{Cilj:} Brisanje korisničkog računa
						\item  \textbf{Sudionici:} Baza podataka
						\item  \textbf{Preduvjet:} Korisnik računa je registriran u sustav
						\item  \textbf{Opis osnovnog tijeka:}
						
						\item[] \begin{packed_enum}
							
							\item Aktor otvara administrativnu ploču
							\item Aktor odabire popis svih registriranih klijenata
							\item Aktor odabire korisnika kojeg želi izbrisati
							\item Aktor briše korisnika
							\item Baza podataka briše račun korisnika
							\item Aktora se vraća na popis svih registriranih klijenata
							\end{packed_enum}
						
					\end{packed_item}
					
				\subsubsection{Dijagrami obrazaca uporabe}
					
					\textit{Prikazati odnos aktora i obrazaca uporabe odgovarajućim UML dijagramom. Nije nužno nacrtati sve na jednom dijagramu. Modelirati po razinama apstrakcije i skupovima srodnih funkcionalnosti.}
				\eject		
				
			\subsection{Sekvencijski dijagrami}
				
				\textbf{\textit{Obrazac uporabe UC2 - Prijava u sustav}}\\
				
				\textit{Korisnik unosi svoje osobne podatke. Sustav provjerava u bazi podataka jesu li uneseni podaci ispravni. Ako jesu, korisnik se prijavi u sustav. No ako uneseni podaci ne odgovaraju niti jednom registriranom korisniku u bazi, sustav korisniku šalje obavijest da su osobni podaci neispravni i daje mu priliku za ponovnu prijavu.}
				\begin{figure}[H]
					\includegraphics[width=\textwidth]{slike/sekvencijski_dijagram_UC2.PNG} %veličina u odnosu na širinu linije
					\caption{Slika broj: Sekvencijski dijagram za UC2}
					\label{fig:UC2} %label mora biti drugaciji za svaku sliku
				\end{figure}
				\clearpage

				\textbf{\textit{Obrazac uporabe UC9 - Pregled skeniranih dokumenata}}\\
				
				\textit{Korisnik odabire pregled popisa svih dokumenata koje je skenirao. Sustav pretraži u bazi njegove skenirane dokumente i vraća ih korisniku na uvid.}
				\begin{figure}[H]
					\includegraphics[width=\textwidth]{slike/sekvencijski_dijagram_UC9.PNG} %veličina u odnosu na širinu linije
					\caption{Slika broj: Sekvencijski dijagram za UC9}
					\label{fig:UC9} %label mora biti drugaciji za svaku sliku
				\end{figure}
				\clearpage

				\textbf{\textit{Obrazac uporabe UC21 - Digitalizacija dokumenta}}\\
				
				\textit{Korisnik šalje sliku  u sustav koji zatim tu sliiku učitava i pregledava je li na slici dokument. Ako je, onda na njega primjenjuje OCR te novonastali dokument sprema u bazu podataka. Ako na slici nije očitan dokument, sustav korisniku šalje obavijst da učitana slika ne sadrži dokument te mu daje opciju da priloži novu sliku.}
				\begin{figure}[H]
					\includegraphics[width=\textwidth]{slike/sekvencijski_dijagram_UC21.PNG} %veličina u odnosu na širinu linije
					\caption{Slika broj: Sekvencijski dijagram za UC21}
					\label{fig:UC21} %label mora biti drugaciji za svaku sliku
				\end{figure}		
				\clearpage

				\eject
	
		\section{Ostali zahtjevi}
		
			\textbf{\textit{dio 1. revizije}}\\
		 
			 \textit{Nefunkcionalni zahtjevi i zahtjevi domene primjene dopunjuju funkcionalne zahtjeve. Oni opisuju \textbf{kako se sustav treba ponašati} i koja \textbf{ograničenja} treba poštivati (performanse, korisničko iskustvo, pouzdanost, standardi kvalitete, sigurnost...). Primjeri takvih zahtjeva u Vašem projektu mogu biti: podržani jezici korisničkog sučelja, vrijeme odziva, najveći mogući podržani broj korisnika, podržane web/mobilne platforme, razina zaštite (protokoli komunikacije, kriptiranje...)... Svaki takav zahtjev potrebno je navesti u jednoj ili dvije rečenice.}
			 
			 
			 
	