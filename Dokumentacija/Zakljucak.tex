\chapter{Zaključak i budući rad}
		
		{Projektni zadatak bio je razviti prilagođeno rješenje za digitalizaciju i baratanje digitaliziranim dokumentima. Glavne ciljne značajke bile su mogućnost digitalizacije određenih formata dokumenata, razvrstavanje u definirane kategorije, mogućnost rada sa spremljenim dokumentima na razini različitih zaposlenika, u što ulazi revidiranje, arhiviranje i potpisivanje te slanje dokumenata putem društvenih mreža. Dodatno se podrazumijeva sustav upravljanja korisnicima . Sve glavne značajke uspješno su implementirane te je provedeno testiranje pojedinih komponenti i testiranje sustava.}
		
		{Na projektnom zadatku radio je studentski tim u efektivnom razdoblju od deset radnih tjedana. Razvojno razdoblje se može sagledati u dva jednaka dijela. Prvo dio odnosio se prvenstveno na proučavanje i dokumentiranje zahtjeva te osmišljanje, razradu i modeliranje rješenja. Cilj je bio izraditi stabilnu bazu na kojoj se dalje moglo raditi u sklopu projekta te sagledati moguće buduće probleme. U sklopu prvog dijela projekta fokus većine tima bio je na proučavanju i izradi valjane dokumentacije te savladavanju osnova odabranih tehnologija. Osnovne funkcionalnosti idejnog rješenja također su implementirane u sklopu prvog dijela. Glavni cilj drugog dijela razvojnog razdoblja bila je puna implementacija idejnog rješenja, dorada izvedenih značajki, testiranje te finalizacija dokumentacije. U drugom dijelu članovi su paralelno nastavili učiti odabrane tehnologije i koristili naučeno pri izradi rješenja. Tijekom rada nisu zabilježeni veći problemi u izradi i implementaciji, niti se razvojni tim susreo s problemima koje nisu mogli riješiti. Kao veći problem može se spomenuti nenadan nedostatak resursa odabranog okruženja na koje je postavljen poslužiteljski dio aplikacije zbog čega se nije moglo koristiti ostvareno OCR rješenje, no tim je efektivno to riješio povezivanje s vanjskim servisom za vršenje OCR procesa nad kojim se zatim vrše dodatne operacije za prilagodbu.}
		
		{S obzirom na to da je razvojni tim uložio mnogo vremena i truda u razradu svih dijelova projekta smatramo kako ne bi bilo problema u slučaju nastavka rada na projektu te bi produktivnost bila na izrazito visokoj razini zbog svega naučenog. Također smatramo kako je projekt u cjelini vrlo dobro postavljen s dovoljno opširnom dokumentacijom što bi omogućilo laku i efektivnu primopredaju drugom razvojnom timu radi nastavka razvoja i dorade rješenja.}
		
		{Razvojni tim smatra kako je vrlo dobro iskoristio svoje vrijeme pri radu na projektu te kako su svi članovi tima imali priliku naučiti nešto novo i razviti se kao budući inženjeri računarstva. Rad na projektu pridonio je poboljšanju komunikacijskih i organizacijskih vještina kod članova, stavio ih u nove situacije gdje su morali naučiti koristiti nove tehnologije, kako za sam razvoj aplikacije već i za izradu dokumentacije te organizaciju razvojnog tima. Razvojni tim smatra kako je projekt završio uspješno te je zadovoljan postignutim rezultatom.}
		
		\eject 